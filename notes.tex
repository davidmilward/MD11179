capture the essentials of the 11179 standard in widely-accepted
language

express the relevant components of that language using a formal set
theoretic notation

key characteristic use cases 
 - metadata about datasets
 - six aims (reducible to three?) in metadata curation 

show that a generalised version exists that has some notion of
closure? 

 * the re-use of value domains 

----------------------------------------------------------------------
1. 11179 design in UML/eCORE
2. 11179 objectives 
----------------------------------------------------------------------


Section 3: relevant aspects of 11179 (could have their UML diagrams,
but the main thing is a handful of choice extracts from the standard -
enough to make it clear what it is that we are looking at)

Section 4: UML description of 11179 including logical constraints 
 
11179-3 context class (p. 56 of Part 3) - scope vs context 

components 

----------------------------------------------------------------------
model driven engineering
 - COMBINING DIFFERENT ASPECTS TO DELIVER PRODUCT (WITH COMPONENTS)
 - COMPOSITIONALITY OF MODELS/DATA ELEMENTS IS ESSENTIAL 

model driven engineering is supporting across product 
 - development continues through deployment 
 - VERSIONING should be about propagating the delta 
----------------------------------------------------------------------

argue that aims - standardisation - this implies that you have to
support change 

----------------------------------------------------------------------

lacks formal, clear account of what is actually to be achieved 
for example - information about the registration authority 

against its own principles really
setting a fixed way of describing a registration authority 

----------------------------------------------------------------------

standard naming of components so that they can identify 

----------------------------------------------------------------------

what do the aims mean in the context of model-driven engineering 
 - references to 11179 literature 
 - references to software engineering/ model-driven engineering 

----------------------------------------------------------------------

agree completely with I 5.1 

5.2.1 the meanings of all the concepts used to describe a datum are
combined into a story, sometimes called a fact 

p.15 ISO/IEC/TR 20943-3 registration and management of value domains

----------------------------------------------------------------------

object class + property 

this is the default filing system for metadata registry contents 

----------------------------------------------------------------------

PURPOSES
— standard description of data;
— common understanding of data across organizational elements and between organizations;
— re-use and standardization of data over time, space, and applications;
— harmonization and standardization of data within an organization and across organizations;
— management of the components of descriptions of 

----------------------------------------------------------------------

e 4 represents several fundamental facts about the four classes:
— A data element is an association of a data element concept and a representation (primarily a
value domain);
 — Many data elements may share the same data element concept, which means a DEC may be represented in many different ways;
— Data elements may share the same representation, which means that a value domain can be reused in other data elements;
— Value domains do not have to be related to a data element and may be managed independently;
— Value domains that share all the value meanings of their permissible
values are conceptually equivalent, so share the same conceptual
domain;

There is one important rule the Figure 4 does not depict: Given a data element, one of the conceptual domains related to its data element concept shall be the conceptual domain of its value domain.
Many other facts are not illustrated in Figure 4, but some of these
are described in Clause 6. 

----------------------------------------------------------------------

Part 1 doesn't have a notion of compliance itself - that comes from
the other parts. 

----------------------------------------------------------------------

