\documentclass{llncs}


\usepackage{listings}
\usepackage{csquotes}
\usepackage{color}
\usepackage{caption}
\usepackage{graphicx}
\usepackage{zed}
\DeclareGraphicsExtensions{.pdf,.png,.jpg}
%%\newtheorem{definition}{Definition}
 \usepackage{listings}
 \usepackage{courier}
 \lstset{
         basicstyle=\footnotesize\ttfamily, % Standardschrift
         %numbers=left,               % Ort der Zeilennummern
         numberstyle=\tiny,          % Stil der Zeilennummern
         %stepnumber=2,               % Abstand zwischen den Zeilennummern
         numbersep=5pt,              % Abstand der Nummern zum Text
         tabsize=2,                  % Groesse von Tabs
         extendedchars=true,         %
         breaklines=true,            % Zeilen werden Umgebrochen
         keywordstyle=\color{red},
    		frame=b,         
 %        keywordstyle=[1]\textbf,    % Stil der Keywords
 %        keywordstyle=[2]\textbf,    %
 %        keywordstyle=[3]\textbf,    %
 %        keywordstyle=[4]\textbf,   \sqrt{\sqrt{}} %
         stringstyle=\color{white}\ttfamily, % Farbe der String
         showspaces=false,           % Leerzeichen anzeigen ?
         showtabs=false,             % Tabs anzeigen ?
         xleftmargin=17pt,
         framexleftmargin=17pt,
         framexrightmargin=5pt,
         framexbottommargin=4pt,
         %backgroundcolor=\color{lightgray},
         showstringspaces=false      % Leerzeichen in Strings anzeigen ?        
 }
 \lstloadlanguages{
         Java
 }
%\DeclareCaptionFont{blue}{\color{blue}} 

 %\captionsetup[lstlisting]{singlelinecheck=false, labelfont={blue}, textfont={blue}}
 % \usepackage{caption}
 
\DeclareCaptionFont{white}{\color{white}}
\DeclareCaptionFormat{listing}{\colorbox[cmyk]{0.43, 0.35, 0.35,0.01}{\parbox{\textwidth}{\hspace{15pt}#1#2#3}}}
 % \captionsetup[lstlisting]{format=listing,labelfont=white,textfont=white, singlelinecheck=false, margin=0pt, font={bf,footnotesize}}



\captionsetup[lstlisting]{format=listing,labelfont=white,textfont=white, singlelinecheck=false, margin=0pt, font={bf,footnotesize}}



\begin{document} 

\title{Metadata Registry and management based on ISO11179 using Model Based Engineering}
%If Title is too long, use \titlerunning
%\titlerunning{Short Title}

%Single insitute
\author{David Milward \and Firstname Lastname}
%If there are too many authors, use \authorrunning
%\authorrunning{First Author et al.}
\institute{University of Oxford}
\maketitle

\begin{abstract}
n this paper we present an ISO11179 metadata registry using a data-oriented Domain Specific Modelling Language(DSML). In particular we examine how certain aspects of the ISO11179 specification can be strengthened by using a specific DSML built to handle interoperability use cases, and also how using a model based engineering framework addresses ambiguities in the standard. We examine how the DSML approach taken in this paper presents a concrete realisation of data componentisation, harmonisation, standardisation and reuse of meta-data components. We also examine how the ISO11179 based DSML can be implemented using the Eclipse Modelling Framework and made interoperable with UML In particular, we identify how Model Driven Engineering has helped in achieving the specific goals of ISO11179 via a case study.

\end{abstract}

\keywords{...}

\noindent

\section{Introduction}

ISO11179 is the ISO standard for metadata registries. Metadata registries are used in many organisations to carry out a number of functions, nearly all of them are related to the need to ensure that data is used consistently within an organization.  The need for such a toolkit has become apparent in the last 10 years or so as the amount of data available to organizations has exploded, and despite the existence of an international standard metadata registries are implemented in a variety of different ways. In this paper we look at the intentions of the standard, and since there is no reference implementation of the standard we attempt to build an ISO11179 metadata registry using model driven engineering principles. During this process we examine the strengths and weaknesses of the standard and highlight areas in which the standard can be strengthened, made more user-friendly, adaptable and workable within an enterprise architectural framework. 


\subsection{The Purpose of ISO11179}

The ISO11179 Standard for metadata registries defines its purpose[REF]as follows,
\newline
to promote:
\begin{itemize}
\item Standard description of data
\item Common understanding of data across organizational elements and between organizations
\item Re-use and standardization of data over time, space, and applications
\item Harmonization and standardization of data within an organization and across organizations
\item Management of the components of data
\item Re-use of the components of data.
\end{itemize}

Interoperability isn't specifically mentioned, however these six items are very close to being a description of interoperability for data and data components. There is no other international standards which tackle the issues of interoperability, although there are a number of accepted ``'maturity models' which target interoperability (REF). These are similar in structure, address similar issues, but diverge slightly on implementation routes. The MDI model framework which emerged from the Athena and Interop NoE (REF) research projects have made progress in defining ways of implementing interoperability using model driven engineering concepts and ideas. Since ISO11179 is currently in use in both the Healthcare and Defence sectors it seems sensible to see if we can apply model driven interoperability concepts to implement the main features detailed in ISO11179, especially since no reference implementation of the standard is provided. It is also a chance to explore how these six use-cases are implemented by the standard and where such implementation is augmented by the use of Model Based Engineering techniques. These core ISO11179 use cases are examined in the next few paragraphs.

 
\subsubsection{Standard Data Description}

The standard data description in ISO11179 is given in section1.6.1 where the idea of a data element is introduced, and with it the idea that it is composed of two parts, its conceptual part and its representational part. This section further puts forward the notion that a data element concept can be composed of two parts, an object class and a property. An object class is said to correspond with a class (in OO terms) or an entity(in ER terms). This is further illustrated with the illustration copied below in figure~\ref{fig:DEC}:

\begin{figure}[h]
\includegraphics[width=0.6\textwidth,natwidth=610,natheight=642]{figs/DataElementConcept}
\caption{Data Elements and Data Element Concepts} 
\label{fig:DEC}
\end{figure}

The standard continues to describe the relationship between data elements and the concepts associated with them, and also puts forward the notion that a data element is produced when a data element concept is associated with a representation. The notion of \emph{value domains} is introduced, where a value domain is defined as:

\begin{quotation} {sets of permissible values for data} \end{quotation}

  this definition is not very far from the online dictionary of computing definition for the notion of \emph{type}: 
\begin{quotation}
\textbf{type} or \textbf{data type} : A set of values from which a variable, constant, function, or other expression may take its value. A type is a classification of data that tells the compiler or interpreter how the programmer intends to use it. For example, the process and result of adding two variables differs greatly according to whether they are integers, floating point numbers, or strings. 
\end{quotation}

This section then looks in detail at ways in which the conceptual aspects of data elements and value domains are related, and during these discussions the a fundamental model of value domains is presented. Other concepts such as measurement units and enumerations are used to further define the role of the value domain within the standard. Section 1.6 discusses aspects of classification of data elements, apart from the previously introduced notions of object class and property which are part of the Data Element Concept idea. An overview of a metadata registry is introduced in section 1.7, the main feature being that it is a database for metadata built along the lines of the conceptual model provided in section 3 of the standard. Section 1.8 covers the rest of the standard, in addition there is a detailed treatment of terminological principles in the appendix.



\subsubsection{Common understanding of Data}

Part 4 of the standard details how to formulate good data definitions, and data definitions are one of the aspects to achieving a common understanding. The advice in this part of the standard can be summarised as follows: 
\begin{itemize}
\item State the essential meaning of the concept
\item Be precise and unambiguous
\item Be concise
\item Be able to stand alone
\item Be expressed without embedding rationale, functional usage, domain information, or procedural information.
\item Avoid circular reasoning
\item Use the same terminology and consistent logical structure for related definitions
\item Be appropriate for the type of metadata being defined.
\end{itemize}
Use of these guidelines will of course help with the development of a common understanding of data, however it is only when these definitions can be viewed alongside the data item in question that these guidelines can be seen to be useful.


\subsubsection{Re-use and Standardization of Data over time, space and applications}

According to section 5 Standardization involves standardizing the descriptive data itself: characteristics, property values of characteristics, selection of signifiers, and the meaning of values.  It can occur at a variety of levels: agency, national , regional, or international. Governance is further applied by accredited standards organizations, who may well have their own standardization processes.


\subsubsection{Harmonization and Standardization of data}


\subsubsection{Management of Data Components}


\subsubsection{Re-use of Data Components}






\section{Domain Specific Modelling Language}

\subsection{Language Definition}

\subsection{Semantics}


\subsection{ISO11179}



\section{Metadata Registry}

\subsection{}


\section{Metadata Management}



\section{Results}

\subsection{Comparison of Metadata Registries}


\section{Discussion}

 


\newpage

\bibliographystyle{plain}

\bibliography{md11179}


\end{document}  