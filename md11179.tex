\documentclass{llncs}

\usepackage{listings}
\usepackage{csquotes}
\usepackage{color}
\usepackage{caption}
\usepackage{graphicx}
\usepackage{zed}
\DeclareGraphicsExtensions{.pdf,.png,.jpg}
 \usepackage{listings}
 \usepackage{courier}
 \lstset{
         basicstyle=\footnotesize\ttfamily, % Standardschrift
         %numbers=left,               % Ort der Zeilennummern
         numberstyle=\tiny,          % Stil der Zeilennummern
         %stepnumber=2,               % Abstand zwischen den Zeilennummern
         numbersep=5pt,              % Abstand der Nummern zum Text
         tabsize=2,                  % Groesse von Tabs
         extendedchars=true,         %
         breaklines=true,            % Zeilen werden Umgebrochen
         keywordstyle=\color{red},
    		frame=b,         
 %        keywordstyle=[1]\textbf,    % Stil der Keywords
 %        keywordstyle=[2]\textbf,    %
 %        keywordstyle=[3]\textbf,    %
 %        keywordstyle=[4]\textbf,   \sqrt{\sqrt{}} %
         stringstyle=\color{white}\ttfamily, % Farbe der String
         showspaces=false,           % Leerzeichen anzeigen ?
         showtabs=false,             % Tabs anzeigen ?
         xleftmargin=17pt,
         framexleftmargin=17pt,
         framexrightmargin=5pt,
         framexbottommargin=4pt,
         %backgroundcolor=\color{lightgray},
         showstringspaces=false      % Leerzeichen in Strings anzeigen ?        
 }
 \lstloadlanguages{
         Java
 }
%\DeclareCaptionFont{blue}{\color{blue}} 

 %\captionsetup[lstlisting]{singlelinecheck=false, labelfont={blue}, textfont={blue}}
 % \usepackage{caption}
 
\DeclareCaptionFont{white}{\color{white}}
\DeclareCaptionFormat{listing}{\colorbox[cmyk]{0.43, 0.35, 0.35,0.01}{\parbox{\textwidth}{\hspace{15pt}#1#2#3}}}
 % \captionsetup[lstlisting]{format=listing,labelfont=white,textfont=white, singlelinecheck=false, margin=0pt, font={bf,footnotesize}}
 
    \graphicspath{{figs/}}
 


\captionsetup[lstlisting]{format=listing,labelfont=white,textfont=white, singlelinecheck=false, margin=0pt, font={bf,footnotesize}}




% we define a set of macros for constants of type 'Kind' 

\def\datamodel{\mathsf{datamodel}}
\def\dataclass{\mathsf{dataclass}}
\def\dataelement{\mathsf{dataelement}}
\def\enum{\mathsf{enum}}
\def\enumeration{\mathsf{enumeration}}
\def\primitivetype{\mathsf{primitivetype}}
\def\datatype{\mathsf{datatype}}

% and for multiplicity 

\def\optional{0{\upto}1}
\def\mandatory{1{\upto} 1}
\def\many{0{\upto}*}

% our partial ordering on constraints

\def\Cimplies{\mathrel{\implies_c}}
\def\Ciff{\mathrel{\iff_c}}

% our partial ordering on text (this will become interesting later)

\def\Timplies{\mathrel{\implies_t}}
\def\Tiff{\mathrel{\iff_t}}

% and conjunction 

\def\Tand{\mathrel{\land_t}}
\def\TAnd{\mathop{\land_t}}

% our globalised version of the defining relations

\def\refines{\mathrel{refines}}
\def\newVersionOf{\mathrel{newVersionOf}}
\def\extends{\mathrel{extends}}
\def\contains{\mathrel{contains}}

% we may have \sqsubseteq and \gg when it comes to analysis 

% our two status values 

\def\draft{\mathsf{draft}}
\def\final{\mathsf{final}}

\usepackage{zed}

\begin{document}

\title{Metadata Registry and management based on ISO11179 using Model Based Engineering}
%If Title is too long, use \titlerunning
%\titlerunning{Short Title}

%Single institute
\author{Jim Davies, David Milward \and Seyyed Shah}
%If there are too many authors, use \authorrunning
%\authorrunning{First Author et al.}
\institute{University of Oxford}
\maketitle

\begin{abstract}
n this paper we present an ISO11179 metadata registry using a data-oriented Domain Specific Modelling Language(DSML). In particular we examine how certain aspects of the ISO11179 specification can be strengthened by using a specific DSML built to handle interoperability use cases, and also how using a model based engineering framework addresses ambiguities in the standard. We examine how the DSML approach taken in this paper presents a concrete realisation of data componentisation, harmonisation, standardisation and reuse of meta-data components. We also examine how the ISO11179 based DSML can be implemented using the Eclipse Modelling Framework and made interoperable with UML In particular, we identify how Model Driven Engineering has helped in achieving the specific goals of ISO11179 via a case study.

\end{abstract}

\keywords{...}

\noindent

\section{Introduction}

ISO11179 is the ISO standard for metadata registries. Metadata registries are used in many organisations to carry out a number of functions, nearly all of them are related to the need to ensure that data is used consistently within an organization.  The need for such a toolkit has become apparent in the last 10 years or so as the amount of data available to organizations has exploded, and despite the existence of an international standard metadata registries are implemented in a variety of different ways. Since the area of Model Driven or Model Based Engineering is based on the idea of describing key abstractions about both the structure and behaviour of data it would seem a logical step to try and build a metadata registry incorporating MBE principles.  We have built a metadata registry and tested within a number of projects in the Healthcare domain. During this process we examined the strengths and weaknesses of the standard and present the results of our research in this paper. 


\section{Backgound}

\subsection{Model Based Engineering}
Model Based Engineering or Model Driven Engineering is the software engineering practise which utilises \emph{Models} as first class entities in analysing and designing software artefacts; code is generated directly from the models, updates are made directly to the models and the code is then regenerated. A situation is achieved whereby a complete round-trip is possible from the model to the code, and then from the code back to exactly the same model. Full round tripping is not always achieved in practise, and sometimes the modelling simply serves as an initial design, from which a shell is generated and the code is completed by software developers. However considerable gains in code generation and prototyping can be achieved, and these make the modelling process worthwhile.

In MBE the overall architecture of a software project is often classified using the notion of abstraction layers. Abstraction is used in many design processes, in software engineering many sub-processes lend themselves to various kinds of mathematical modelling.  In object oriented design objects are used as the main components of a program, rather than sub-routines or functions.  These objects can be \emph{abstracted} to a kind of templated object called a \emph{class}, a class will contain a blueprint for both data and methods.  Using this idea we consider the the program running, that is the interaction of different objects, to be at level M0.  At this level the program will consist of different objects interacting according original program code, which is written for the most part a collection of various \emph{classes}.  The program code, in effect the blueprint for the program, is considered to be an abstraction of the running program, and is considered to be at level M1.  A model or program at level M1 is defined by a blueprint of model, in effect a \textbf{meta-model} which exists that level M2, and likewise a \textbf{meta-model} can be defined at a higher level of abstraction.  This abstraction \emph{layering} can continue ad-infinitum, however most MBE practitioners use 4 levels, indeed the Model Driven Architecture agreed and specified by the OMG \cite{MOF242}, also known as the Meta Object Facility (MOF) currently limits itself to 4 layers, which are illustrated in Figure~\ref{fig:mbe1}

\begin{figure}[h]
\includegraphics[width=1.0\textwidth,natwidth=610,natheight=642]{Models1}
\caption{Abstraction Layers in Model Based Engineering} 
\label{fig:mbe1}
\end{figure}

We can see that UML itself sits at what is refered to as the M2 level in this architecture model. Each abstraction layer is defined so that the level \emph{below} can be seen as being an implementation of that layer; in this way the UML model is one implementation out of many possible implementations of the UML language. The UML Language itself is an implementation of the MOF meta-meta-language which resides at the M3 layer. It is possible to use UML to define other meta-models at the M2 level, one could for instance use a sub-set of UML to define just a data language at this level.

When one builds a model for a system using MBE the model is at the same level of abstraction as the program, and can be seen as different way of representing the same thing.  A Java or C\# program therefore would sit at the M1 level, and would have a direct one-to-one relationship with a UML model at that level. The idea being that by building the model in say UML one can automatically generate the code in java.

\subsubsection{Domain Specific Languages(DSL)}
The term \emph{Domain Specific Language} is used to talk about a language which has been built purely to be used within one narrow domain of concern. It is more specialized than a general purpose language, but will generally make writting program code within that narrow domain easier.  An example here would be the language \emph{Gradle} which has been built to solve the problem of managing software development projects, and uses as first class citizens the nouns and verbs of the software development process, such as \emph{clean, run, process, etc}. It allows a domain expert to write a build process from scratch without having to know how to do a lot of the detailed work in the programming language. Gradle is written in Groovy, and compiles to the JVM, however it used by software developers developing in both JVM and non-JVM languages alike.

The term DSL is a general term and covers many different types of language.  For our purposes here we consider two main aspects of software analysis and design, the first is the modelling of the domain and the second is engineering an efficient programming language to express these models.These two aspects are often merged, however the two aspects are sometimes treated separately as with the layered modelling approach taken by Latry et al~\cite{latry2006processing}. 


\subsubsection{Domain Specific Modelling Language(DSML)}

A Domain Specific Modelling Language is one that is purely targeting the problem domain, in that it describes problems and solutions in domain terms, and provides a language which domain experts can understand and implement within their domain. In this regard we are considering the domain of managing large datasets, ontologies and classification systems, and the domain experts we are considering are those who will create and edit these datasets. 

\subsubsection{Domain Specific Programming Language(DSPL)}	
A Domain Specific Programming language is concerned with how to implement a suitable language using a general purpose programming language(GPL), frameworks, aspects, etc. There are no real rules to developing a DSPL, and it maybe that developing a DSPL is simply a stage in the development of the DSL, or it may be a module within the DSL. For the most part we haven't defined a DSPL for this project, the implementation has been carried out using the Groovy language. 

\subsubsection{Platform Independent Implementation(PIM)}

**********
\subsubsection{Platform Specific Implementation(PIM)}

\subsection{ISO11179}

The ISO11179 Standard for metadata registries defines its purposes (ISO11179-1 section 0.2 General description of ISO/IEC 11179)as follows,
\newline
to promote:
\begin{itemize}
\item Standard description of data
\item Common understanding of data across organizational elements and between organizations
\item Re-use and standardization of data over time, space, and applications
\item Harmonization and standardization of data within an organization and across organizations
\item Management of the components \emph{of descriptions} of data
\item Re-use of the components \emph{of descriptions} of data.
\end{itemize}
ISO/IEC 11179 is in effect a standard for metadata-driven exchange of data in an heterogeneous environment, based on exact definitions of data. Interoperability isn't explicitly mentioned as aan aim, however these six aims/purposes are very close to being a description of a framework for interoperability for data and data components through the use of \emph{metadata}. There is no international standard which specifically addresses interoperability, although there are a number of accepted maturity models which address interoperability issues(NIEM, ECInterop) within the enterprise.  %%The MDI model framework which emerged from the Athena and Interop NoE (INTEROP, Athena) research projects have made progress in defining ways of implementing interoperability using model driven engineering concepts and ideas, and since ISO11179 is currently in use in both the Healthcare and Defence sectors we examine the core purposes of the standard to try and determine how we it achieves these purposes. %%
The standard itself is not a specification for building a physical or logical metadata registry, but rather a set of semantic principles on how data relationships should be handled. We implement an ISO11179 conformant metadata registry using Model Based Engineering principles, and examine how use of these principles have helped achieved the purposes of ISO11179. 




\section{Design of an ISO11179 Metadata Registry}

The ISO11179 standard as discussed describes a metamodel using text and UML 2.4. In order to arrive at a working model based specification we have taken that description and written it using a formal specification language.  This enabled us to generate both the registry and some of the transformations required to express datasets in terms of the standard.  ISO11179-3 has a detailed account of the registry metamodel and its attributes, and sets out to be relevent to application designers, system architects, and software developers. It uses UML 2.4.1 as a modelling language to describe the main features of a conformant metadata registry, although the standard itself is quite clear in not endorsing any particular environment, database management system, database design paradigm, system development methodology, data definition language, command language, system interface, user interface, computing platform or technology required for implementation. However in ISO11179-3.5 a metamodel is used to describe the information model of a metadata registry, and again the standard is clear that this should not limit the actual implementation technology used. The UML description is split into a number of packages:
\begin{itemize}
\item Basic Package
\item Identification Metamodel
\item Designation and Definition Metamodel
\item Registration Package
\item Concepts Package (Concepts and Classification regions)
\item Binary Relations Package
\item Data Descriptions Package
\end{itemize}

The standard continues to provide a detailed account of how metadata items are related within the metadata registry, for the purposes of brevity we will not repeat them all here, but we may refer back to specific parts of the standard where required.

\subsection{Registry Overview}

The metamodel or language for data we describe is informed by experience with Healthcare information systems \cite{DSMCR} and by the development of an open source Metadata Registry built around similar principles. The DSML has been built to help with data curation and creation in the Healthcare domain, and as such is the result of discussion with domain experts in this sector. The reason for designing this DSML  is that we seek a meta-language to be able handle hetrogeneous metadata stored using a variety of languages. It needs to be able to capture metadata, in particular a detailed description the data is stored in relational databases, XML Schemas, XML files based on XML Schemas, Excel files and object-oriented programs. It then needs to be able to automatically transform the data into a number of simple formats. To do this each dataset at an M1 level needs to be described at the M2 level, by doing this we can build a container which can hold the metadata collected at the M1 level.

The core entities in the metamodel are as follows:
\begin{itemize}
\item DataModel
\item DataClass
\item DataItem
\item DataType
\item DataConstraint
\end{itemize}

\begin{figure}[h]
\includegraphics[width=1.0\textwidth,natwidth=610,natheight=642]{LemmaCore1}
\caption{Core Metamodel (Lemma)} 
\label{fig:lemma}
\end{figure}

Since all items are in effect \emph{ConceptItems} they can be considered to be representing a concept; furthermore all such items can have a relationship with any other conceptitem, and can be \emph{tagged} to reference items outside the registry, such as IRI's or webpages. Some of the core items in ISO11179-3 which are represented by UML diagrams can be directly represented in the core language, others are built in the manner described in the UML diagrams. 

The CaBIG registry was built on similar principles, but required a mapping algorithm to translate from UML to the ISO11179 Conceptual model as is documented in the paper by Kunz et al~\cite{Kunz2009}. In fact several mapping algorithms were tested, the main mapping is shown in Figure~\ref{fig:mapping}. By using a model based approach we avoid the need for such a mapping, instead our registry simply defines an ISO11179 view on the various datasets which are imported into the registry.  Some mapping algorithms are used in mapping XML to this model, and in comparing different dataitems within the registry.

\begin{figure}[h]
\includegraphics[width=1.0\textwidth,natwidth=610,natheight=642]{ISOUMLTransform}
\caption{CaBIG ISO to UML Tranform} 
\label{fig:mapping}
\end{figure}



\subsection{Registry Contents}



All of our conceptitems in the registry will have identifiers, unique within the context of the registry, this implements Rule 1 of the previous section.  Classes will be uniquely named within models, elements will be uniquely named within classes.  
\begin{zed}
  [Id, Name]
\end{zed}

A data element has multiplicity: the three values that are important
to us are as follows:
\begin{syntax}
  Multiplicity & ::= & \optional \mid \mandatory \mid \many 
\end{syntax}

It has also ordering 
\begin{syntax}
  Ordering & ::= & ordered \mid notOrdered 
\end{syntax}

Although a registry could support other kinds of link, there are two that are important to us here: $extends$ and $newVersionOf$. These refer to other items, using identifiers rather than names. The intended semantics of an conceptitem consists primarily in its textual explanation.  Additional information comes from any other items that contain the element: the class, and the model. We introduce the \emph{Interpretation}.
\begin{zed}
  [Interpretation]
\end{zed}
A piece of text can be interpreted in multiple ways, and thus: 
\begin{zed}
  Text == \power Interpretation
\end{zed}

The language is a language of \emph{metadata} and it will normally sit within the context of a metadata registry or similar server artefact. The core entity in our new model is the \emph{ConceptItem} which is in effect an abstract entity from which most of the other entity derive. At its core is the idea that it represents a concept in some shape or form, and that we can make the name and text description originate from a vocabulary or terminology server. As an entity it can be related to any other entity in the registry using the Relationship artefact. Versioning control is handled through the use of a GUID, which has the added benfit of turning metadata into \emph{linked metadata} automatically. The Tag class allows any vocabulary or ontology element to be associated with any catalogue element. This ensures the import of standard ontologies into a metadata registry built around this metamodel.

\begin{schema}{ConceptItem}
  id : Id \\
  name : Name \\
  text : Text \\
  kind : Kind \\  
  extends, newVersionOf : \power Id
\end{schema}

\subsubsection{DataModel}

 DataModel is a versioning and grouping mechanism for one metadata-set, it might for instance be metadata connected with a particular database, or a particular XSD Schema which is used in a particular domain, however it allows us to group and version a related set of DataClasses.  A DataModel can contain many DataElements, and many DataTypes. A DataModel contains one or more DataItems, which may be implemented as DataClasses, DataElements, DataTypes or EnumerationValues. As all DataItems are also ConceptItems, they can have Tags, which link them to entities outside that DataModel using URI's, and they can have Relationships which are two-way associations with other ConceptItems. DataConstraints are attached to each DataItem in the DataModel. A DataModel also declares a status, which can be draft or final.

\begin{syntax}
  Status & ::= & \draft \mid \final 
\end{syntax}

Each datamodel has a globally unique identifier.  
\begin{zed}
  [GUID]
\end{zed}

\begin{schema}{DataModel}
  ConceptItem \\
  guid : GUID \\
  dataclasses, dataelements, datatypes, enumerations, primitivetypes : \power Id \\
  imports : \power Id \\
  status : Status 
  \where
  kind = \datamodel
\end{schema}


\subsubsection{DataItem}
A DataItem is child artefact of a ConceptItem, with the ability to contain a DataConstraint. DataConstraints may be described in the context of any sub-type of a DataItem, that is DataElement, DataType, Enumeration or Enum. Since DataModels contain DataItems one can also group the conjunction of constraint information at the DataModel level.  All that we need to know of DataConstraints is that an implication ordering exists as a partial order upon this given type. 

\begin{zed}
  [DataConstraint]
\end{zed}

%%inrel \Cimplies 
\begin{axdef}
  \_ \Cimplies \_ : DataConstraint \rel DataConstraint 
  \where
  \id DataConstraint \subseteq (\_ \Cimplies \_) \\
  (\_ \Cimplies \_) \cap (\_ \Cimplies \_) \inv = \id DataConstraint   
\end{axdef}



\begin{schema}{DataItem}
 ConceptItem \\
 dataConstraint : DataConstraint \\
\end{schema}


\subsubsection{DataClass}
A DataClass is a mechanism for grouping DataItems, which are atomic pieces of data. A concept may be represented by a DataClass or a DataElement, but the DataItem is the atomic component of that DataClass, it can't be reduced into any smaller component. A DataClass can be used as a DataItem, in that it can be contained within another DataClass, and so it can be used to provide multi-level grouping within a DataModel. A DataClass can \emph{extend} another DataClass, this \emph{inheritance} mechanism is very straightforward since we are only considering structure and not behaviour; it allows the child class to have all the member DataItems and DataClasses that are present in the parent. These DataItems and DataClasses are in effect references, so that if the parent class changes then these \emph{inherited} DataItems and DataClasses will be changed as well.
For the formal specification of the DataClass, we record extension and composition relationships.

\begin{schema}{DataClass}
  DataItem \\
  dataelements : \power Id \\
  extends : \power Id \\
  components : \power Id 
  \where
  kind = \dataclass
\end{schema}

\subsubsection{DataElement}

A DataElement is the smallest data entity described in this langauge, it has a direct one to one relationship with a DataType, since every DataElement will have a corresponding DataType. DataElements have multiplicity and ordering. 

\begin{schema}{DataElement}
  DataItem \\
  multiplicity : Multiplicity \\
  ordering : Ordering \\
  dataType : Id 
  \where
  kind = \dataelement
\end{schema}

\subsubsection{DataTypes}

Enumerations have a sequence of references to values. 

\begin{schema}{Enumeration}
  DataItem \\
  enums : \power Id 
  \where
  kind = \enumeration
\end{schema}

Enumerated values have a value.  

\begin{zed}
  [Value]
\end{zed}

\begin{schema}{Enum}
  DataItem \\
  value : Value 
  \where
  kind = \enum 
\end{schema}

And primitives are primitives. 

\begin{schema}{PrimitiveType}
  DataItem \\
  values : \power Value 
  \where
  kind = \primitivetype
\end{schema}

\subsubsection{Items}

Here we are looking at any language entity which could be present in a registry, an item could be any of the following artefacts, all of which will be kept within a metadata registry.

\begin{zed}
  Item \defs DataModel \lor DataClass \lor DataElement \lor DataType \\
                                            \lor PrimitiveType  \lor Enumeration \lor Enum
\end{zed}

************ Need to include Tag, Relationship, ReferenceType



\subsection{The Registry Itself}

A registry is an indexed collection of items. 

The path to an item starts with the guid of a model, and is then followed by a sequence of names. 
\begin{zed}
  Path == \seq Name 
\end{zed}

\begin{schema}{Concept\_Items}
  item : Id \pfun Item \\
  items : \power Id \\
  path : Id \pfun Id \pfun Path 
\end{schema}

We introduce names for the sets of identifiers pointing to different
kinds of items:

\begin{schema}{Registry\_Sets}
  datamodels, dataclasses, dataelements, enumerations, datatypes, \\
  primitivetype, enums : \power
  Id 
\end{schema}

We introduce also the `global' relationships implied by the reference-valued attributes $extends$ and $newVersionOf$:

%%inrel \refines \newVersionOf
\begin{schema}{Registry\_Links}
  \_ \extends \_, \_ \newVersionOf \_ : Id \rel Id 
\end{schema}

Similarly, we introduce the global relationships implied by containment: 

%%inrel \contains \extends
\begin{schema}{Registry\_Structure}
  \_ \contains \_ : Id \rel Id \\
  datamodel : Id \pfun Id \\
  contents, scope : Id \pfun \power Id \\
  values : Id \pfun \power Value 
\end{schema}

We may now describe registry properties in terms of the combination of all of these identifiers:

\begin{schema}{Registry\_Contents}
  Registry\_Items \\
  Registry\_Sets \\
  Registry\_Links \\
  Registry\_Structure
\end{schema}

\subsubsection{Consistency and derivation}


\begin{schema}{Registry\_Items\_Derivation}
  Registry\_Contents
  \where
  \dom conceptitem = conceptitem \\
  \forall m : datamodels ; i : conceptitems \spot {} \\ \t1 
  \LET c == (\mu x : dataclasses \mid i \in (conceptitem~x).elements) \spot
  {} 
  \\ \t2 
  path~m~i = \langle (conceptitem~m).name, (conceptitem~c).name, (conceptitem~i).name
  \rangle 
\end{schema}

\begin{schema}{Registry\_Sets\_Derivation}
  Registry\_Contents
  \where 
  datamodels = \{ i : \dom conceptitem \mid (conceptitem~i).kind = \datamodel \} \\
  dataclasses = \{ i : \dom conceptitem \mid (conceptitem~i).kind = \dataclass \} \\
  dataelements = \{ i : \dom conceptitem \mid (conceptitem~i).kind = \dataelement \} \\
  enumerations = \{ i : \dom conceptitem \mid (item~i).kind = \enumeration \} \\
  primitivetypes = \{ i : \dom conceptitem \mid (conceptitem~i).kind = \primitivetype \} \\
  enums = \{ i : \dom conceptitem \mid (conceptitem~i).kind = \enum \} \\
  \dom conceptitem = \bigcup \{ datamodels, dataclasses, dataelements, enumerations,
  primitivetypes, enums \} \\
  final = \bigcup \{~i : datamodels \mid (conceptitem~i).status = \final \spot \{ i \}
  \cup contents~i~\} 
\end{schema}

\begin{schema}{Registry\_Links\_Derivation}
  Registry\_Contents 
  \where 
  (\_ \newVersionOf \_) \in datamodels \rel datamodels \\
  (\_ \refines \_) \in {} \\ \t1 
  (dataclasses \rel classes) \cup (dataelements \rel
  dataelements) \cup (enums \rel enums) \\
  \forall i,j : \dom conceptitem \spot {} \\
  \t1 i \newVersionOf j \iff j \in (conceptitem~i).newVersionOf \land {} \\
  \t1 i \extends j \iff j \in (conceptitem~i).extends 
\end{schema}

That is, only models are versioned, and every semantic link connects two items of the same kind, whether these are dataclasses, dataelements, or values.

One item is contained within another if it is declared and managed in the context of that other item: for example, every element is declared in the context of a unique dataclass, and every item is declared in the context of a datamodel.  For convenience, we define not only a $contains$ relation, between references, but also a $contents$ function, returning references to all of the items contained within a referenced item.

The $extends$ function is more straightforward. 

\begin{schema}{Registry\_Structure\_Derivation\_Contents}
  Registry\_Contents
  \where
  \forall m : datamodels \spot contents~m = {} \\
  \t1 (item~m).dataclasses \cup (item~m).enumerations \cup
  (item~m).primitivetypes \cup {} \\
  \t1 \bigcup \{~c : (item~m).dataclasses \spot contents~c~\} \cup {} \\
  \t1 \bigcup \{~n : (item~m).enumerations \spot contents~n~\}
  \\
  \forall c : dataclasses \spot contents~c = {} \\
  \t1 (item~c).dataelements  \cup {} \\
  \t1 \bigcup \{~s : (item~c).components \spot contents~s ~\}
  \\
  \forall n : enumerations \spot contents~n = (item~n).enums
  \\
  \dom contents = \bigcup \{ datamodels,data classes, enumerations \} 
  \\
  \bigcup (\ran contents) = \bigcup \{ dataclasses, dataelements, enumerations, enums,
  primitivetypes \} 
  \\
  \forall i : \dom item \spot contents~i = (\_ \contains \_) ~\limg \{ i \}
  \rimg \\
  (\_ \extends \_) \in dataclasses \rel dataclasses \\
  \forall c, d : dataclasses \spot c \extends d \iff 
  d \in (item~c).extends
\end{schema}

That is, the contents of a model includes the contents of any classes and enumerations that it contains, a class contains its elements, and an enumeration contains its values.  Only models, classes, and enumerations have contents, and only classes, elements, enumerations, values, and primitives---not models---can be contained. 

A related but not equivalent notion is that of the `scope' of a model. This is a set of references to every item contained within that model, or contained within a model that it `imports'. 

\begin{schema}{Registry\_Structure\_Derivation\_Scope}
  Registry\_Contents
  \where
  \dom scope = datamodels 
  \\
  \forall m : datamodels \spot scope~m = contents~m \cup \bigcup 
  ( contents~ \limg (item~m).imports \rimg )
\end{schema}

It will be convenient also to consider the inverse relational image of
the closure of this function.  For any non-model item in the
catalogue, this will return a reference to the unique model containing
that item. 

\begin{schema}{Registry\_Structure\_Derivation\_Model}
  Registry\_Contents
  \where
  \dom model = \bigcup \{ dataclasses, dataelements, enumerations, enums,
  primitivetypes \} \\
  datamodel = ((\_ \contains \_)\star)\inv \rres datamodels
\end{schema}

\begin{zed}
  Registry\_Structure\_Derivation \defs {} \\ \t1 
  Registry\_Structure\_Derivation\_Contents \land {} \\ \t1 
  Registry\_Structure\_Derivation\_Scope \land {} \\ \t1 
  Registry\_Structure\_Derivation\_Model   
\end{zed}

\begin{zed}
  Registry\_Derivation \defs {} \\ \t1 
  Registry\_Sets\_Derivation \land {} \\ \t1 
  Registry\_Links\_Derivation \land {} \\ \t1 
  Registry\_Structure\_Derivation 
\end{zed}

\subsection{Registry Properties}

A datamodel can import only other datamodels known to the registry.

\begin{schema}{Registry\_Imports}
  Registry\_Contents 
  \where
  \forall m : datamodels \spot (item~m).imports \subseteq datamodels
\end{schema}

Every reference to a type must be to a type known to the catalogue: a
known class, enumeration, or primitive. 

\begin{schema}{Registry\_TypesDefined}
  Registry\_Contents
  \where
  \forall e : dataelements \spot (item~e).type \in dataclasses \cup enumerations \cup
  primitives 
\end{schema}

A datamodel may be a new version only of a finalised datamodel.  A datamodel may be final only if all imported datamodel are also final.  An item may be finalised only if all of its outgoing semantic links are to finalised items.

\begin{schema}{Registry\_Final}
 Registry\_Contents
  \where
  \ran (\_ \newVersionOf \_) \subseteq final \\
  \forall f : final \cap datamodels \spot (item~f).imports \subseteq final \\
  (\_ \refines \_) \limg final \rimg \subseteq final 
\end{schema}

\begin{zed}
  Registry\_Properties \defs {} \\ \t1
  Registry\_Imports \land {} \\ \t1 
  Registry\_TypesDefined \land {} \\ \t1
  Registry\_Final 
\end{zed}

\subsection{Semantics}

***************NEEDS REVISION*************************

The `semantics' of a data element is a mapping from possible values to \emph{a set of} possible interpretations.  Or, equally, a \emph{relation} between values and interpretations. 

The intended interpretation of a given value, in the context of a given registry, will depend upon:
\begin{itemize}
\item the text describing the value
\item the text describing the value type (primitive, enumerated or referenced)
\item the text describing the dataelement
\item the text describing the dataclass in which the element is defined,
  and any dataclasses `containing' that dataclass
\item the text describing the datamodel containing the dataelement
\item the text of any item that these items are linked to using$\extends$ 
\end{itemize}
Any of this text---even that which has been attached to a primitive value domain---may have something specific to say about a given data element.  


\section{Metadata Management}
This section gives a brief overview of the tasks which data registration authorities face, the principle problem being what data is held where, what formats are being used and how does it relate to other datasets being managed directly or indirectly. It could be that one data item has a reference to another data item managed by a different authority and the only reference is a written instruction stating that this \emph{dataitem is not the same as that dataitem}. 

The principal tasks faced by a data registration authority are given as:
\begin{itemize}
\item Creation of new datasets
\item Curation of existing datasets
\item Merging of existing datasets
\item Querying over existing datasets
\end{itemize}

The first task or use case is relatively straightforward, however there is one aspect that needs discussion. Very often datasets are developed by teams, and often using tools that are not really designed for dataset development. In the NHS we have found that many datasets are initially developed by domain experts using Excel, mostly because it is accessible and is easy to interpret at a high level. One of the biggest problems encountered here is simply co-ordination amongst the contributors. 

The next task is curation of existing datasets, once a dataset is published and is being used it's shortcomings will be noted by users who will lobby for changes. Datasets are easier to manage if they conform to the Open-Closed Principle of software engineering, first introduced by Bertram Meyer \cite{Meyer} which states that a module should be open to extension and closed to change. This means that datasets or datamodels can be extended by being added to, but that existing items should not be changed. Changing is possible, but it implies the start of a completely new DataModel, since it is very difficult to migrate existing artefacts from DataModel A to DataModel B if something has been removed or changed in A, however the addition of say another DataElement won't break existing artefacts.

The merging of datasets is another area of curation which is carried out by any registration authority and which is help by a metadata registry. The important points to note here are 1) the abiltiy to identify a particular element 2) the ability to relate it to another particular element and 3) the ability to measure the difference and decide which parts go into the merged dataset. Inevitably the mered dataset will contain back references to the originating dataitems, how easy these are to access and address is another point of note for the data curator.

The last point is the querying of data itself, although this is not really a function of a metadata registry, data querying is dependent on having data indexed in a structured manner, even if the data itself is stored in an unstructured manner. The better the index the better the search, and in many cases indexed searches on unstructured data can be fast and more effective that searches on data stored in normalized relational databases. 



\section{Results}




\section{Discussion}

\section{Related Work}

 


\newpage

\bibliographystyle{plain}

\bibliography{md11179}

\end{document}

